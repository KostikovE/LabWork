%https://ru.overleaf.com/learn

% Логика документа

% Выбираем класс документа и классовые опции 
% Класс определяет вид и структуру документа. 
% Класс -- это база, которую можно править с помощью подключаемых стилевых файлов. 
% В классе задаётся геометрия страницы и определены команды секционирования. 

% Имеются следующие базовые классы:
% article (статья), 
% book (книга), 
% report (отчёт), 
% proc (доклад), 
% letter (письмо) и 
% slides (слайды)!.

% Стандартные стилевые опции:
% 10pt|11pt|12pt — установка базового размера шрифта. 
% a4paper — установка размера листа бумаги. Следует использовать всегда, так как по умолчанию LATEX использует размер листа letter.
% draft — режим черновой печати для «отлавливания» проблем вёрстки. 
% oneside|twoside — форматирование документа для односторонней и двухсторонней печати, соответственно.
% twocolumn — печать в две колонки.

% Модификации классов:
% extarticle, extbook, extletter, extproc, extreport (8pt, 9pt, 14pt, 17pt и 20pt)

\documentclass[a4paper, 14pt]{article} 
% Пакеты подключаются командой 
% \usepackage[необязательные параметры]{имя пакета}

% Выбор кодировки файла
\usepackage[utf8]{inputenc}

\usepackage{amsmath, amssymb}

\usepackage{listings}

%Выбор кодировки ширифтов
%для тестов только на русском языке этот параметр можно не указывать
\usepackage[T2A]{fontenc} 
% Подключение языков документа
% Последний язык доминирующий
\usepackage[english, russian]{babel}  

%Задаем параметры документа
\usepackage[top = 20 mm, 
            bottom = 20 mm, 
            left = 30 mm, 
            right = 30 mm]{geometry}
            
%Красная строка в первом абзаце
\usepackage{indentfirst}
%Величина отступа красной строки
\setlength{\parindent}{12.5 mm}

%Межстрочный интервал
%\def\baselinestretch{1.5}
\usepackage{setspace}
\setstretch{1}

% Задание стиля оформления страницы
% empty нет ни колонтитулов, ни номеров страниц;
% plain номера страниц ставятся внизу в середине строки, колонтитулов нет; основной стиль — article
% headings присутствуют колонтитулы (включающие в себя и номера страниц); основной стиль — book
% myheadings присутствуют колонтитулы, оформленные так же, как в предыдущем случае; отличие в том, что текст, печатающийся в колонтитулах (в стандартном случае это номера и названия разделов документа), не порождаются \LaTeX’ом автоматически, а задается пользователем в явном виде.

%Команда \thispagestyle, задающая стиль оформления одной отдельно взятой страницы

\pagestyle{plain}



\usepackage{caption}
\captionsetup[figure]{
                    labelsep=endash, 
                    font=large, 
                    justification=centering, 
                    singlelinecheck=false
                    }

\begin{document}
\fontsize{14}{18}\selectfont
\thispagestyle{empty}
\newpage
\tableofcontents
\newpage

\section{Набор формул}

\begin{center}
{\bf Степени и индексы}
\end{center}

\noindent $\blacktriangleright$ Набор в \LaTeX:
\begin{lstlisting}
$$
R_{i,j}^{k,n}
$$
\end{lstlisting}

\noindent $\blacktriangleright$ На печати 
$$
R_{i,j}^{k,n}
$$

\begin{center}
{\bf Дроби}    
\end{center}

\noindent $\blacktriangleright$ Набор в \LaTeX:
\begin{lstlisting}
$$
\frac{1}{2}, 
\frac{1}{1+\frac{1}{2}}
$$
\end{lstlisting}
\noindent $\blacktriangleright$ На печати 
$$
\frac{1}{2}, \frac{1}{1+\frac{1}{2}}
$$

\noindent $\blacktriangleright$ Набор в \LaTeX:
\begin{lstlisting}
$$
\dfrac{1}{2}, 
\dfrac{1}{1+\dfrac{1}{2}}
$$
\end{lstlisting}
\noindent $\blacktriangleright$ На печати 
$$
\dfrac{1}{2}, \dfrac{1}{1+\dfrac{1}{2}}
$$

\begin{center}
{\bf Скобки переменного размера}    
\end{center}

\noindent $\blacktriangleright$ Набор в \LaTeX:
\begin{lstlisting}
$$
\left.
\left(T
\right) \dfrac{1}{2}
\right)
$$
\end{lstlisting}

$$
\left.
\left(T
\right) \dfrac{1}{2}
\right)
$$



Корни
$$
\sqrt{4}
$$

Штрифи и многоточия
$$
f''
$$
$$
\ldots \cdots \vdots \ddots 
$$

Имена математических функций
$$
\sin() \cos() \tanh \log_{10}{2} \ln
$$
Греческий алфавит
$$\alpha, \beta \Sigma \sigma \epsilon \varepsilon$$
Символы
$$
\diamond
\blacktriangleleft
$$

Операции с пределами и без
$$
\left.\int\limits_{-\infty}^{+\infty} \sin() \, dx = -\cos(x)\right|_{a}^{b}
$$
$$
\sum\limits_{i=1}^{n}
$$

Нумерация формул
\begin{equation} \label{eq2}
\cos(x)    
\end{equation}

\begin{verbatim}
\begin{equation*} 
\sin(x)    \leqno{(**)}
\end{equation*}
\end{verbatim}


\begin{equation*} 
\sin(x) \leqno{(**)}
\end{equation*}

\begin{equation*} 
\cos(x)  \eqno{(12)}  
\end{equation*}

Включение текста в формулы \eqref{eq2}

\begin{center}
Надстрочные символы
\end{center}
$$
\overline{1,k}, \quad
\hat{x} \quad
\widehat{AB} \quad
\overrightarrow{AB}
$$
Для набора матриц используются следующие окружения:
$$
\begin{pmatrix}
a_{11} & a_{12} & \ldots & a_{1n}\\
a_{21} & a_{22} & \ldots & a_{2n}\\
\vdots & \vdots & \ddots & \vdots \\
a_{n1} & a_{n2} & \ldots & a_{nn}
\end{pmatrix}
$$

$$
\begin{vmatrix}
a_{11} & a_{12} & \ldots & a_{1n}\\
a_{21} & a_{22} & \ldots & a_{2n}\\
\vdots & \vdots & \ddots & \vdots \\
a_{n1} & a_{n2} & \ldots & a_{nn}
\end{vmatrix}
$$

\begin{equation*} 
\sin(x)    \leqno{(**)}
\end{equation*}

$$
\left(
\begin{array}{cccc}
a_{11} & a_{12} & \ldots & a_{1n}\\
a_{21} & a_{22} & \ldots & a_{2n}\\
\vdots & \vdots & \ddots & \vdots \\
a_{n1} & a_{n2} & \ldots & a_{nn}
\end{array}
\right)
$$
\parbox{75 mm}{\begin{multline} 
1+2+3+4+5+ \\
+6+7+8+ \\
 +9+10 = 45. 
\end{multline}}
\parbox{75 mm}{\begin{multline} 
1+2+3+4+5+ \\
+6+7+8+ \\
 +9+10 = 45. 
\end{multline}}

Многострочные выключные формулы

\begin{multline*} 
1+2+3+4+5+ \\
+6+7+8+ \\
 +9+10 = 45. 
\end{multline*}

\begin{gather} %\notag
1+2=3, \notag \\
1+4=5, \\
100+101 = 201.
\end{gather}

\begin{align}  %\notag
1+2=3, \\
1+4=5, \notag \\
100+101 = 201.
\end{align}


\begin{equation}
\begin{split}
1999&=1000+900+{}\\
&+90+9
\end{split}
\end{equation}

\begin{align}
7\times 9& =63 & 63:9& =7\\
9\times 10& =90 & 90:10& =9
\end{align}


Пробелы в формулах вручную \\
\begin{tabular}{|c|c|c|}
\hline
Синтаксис в \LaTeX & Комментарий & Примеры \\ \hline
\begin{lstlisting}
$x\quad y$
\end{lstlisting}
& Пробел в 1em
& $x\quad y$ \\ \hline
\begin{lstlisting}
$x\qquad y$
\end{lstlisting}
& Пробел в 2em
& $x\qquad y$ \\ \hline
\begin{lstlisting}
$\int\sin(x)dx$
\end{lstlisting}
& Без пробела
& $\int\sin(x)dx$\\ \hline
\begin{lstlisting}
$\int\sin(x)\!dx$
\end{lstlisting}
& Отрицательный пробел
& $\int\sin(x)\!dx$\\ \hline
\begin{lstlisting}
$\int\sin(x)\,dx$
\end{lstlisting}
& Тонкий пробел
& $\int\sin(x)\,dx$\\ \hline
\begin{lstlisting}
$\int\sin(x)\:dx$
\end{lstlisting}
& Средний пробел
& $\int\sin(x)\:dx$\\ \hline
\begin{lstlisting}
$\int\sin(x)\;dx$
\end{lstlisting}
& Толстый пробел
& $\int\sin(x)\;dx$\\ \hline
\end{tabular}

Листинги

\begin{lstlisting}
\begin{equation}
\begin{split}
1999&=1000+900+{}\\
&+90+9
\end{split}
\end{equation}
\end{lstlisting}

\section{Набор текста}

\section{Верстка таблиц}

\section{Подготовка презентаций}

\end{document}

